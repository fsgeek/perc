% Tony Mason (fsgeek@cs.ubc.ca)
%
% TEMPLATE for Usenix papers, specifically to meet requirements of USENIX '05
% originally a template for producing IEEE-format articles using LaTeX.
%
% written by Matthew Ward, CS Department, Worcester Polytechnic Institute.
% adapted by David Beazley for his excellent SWIG paper in Proceedings,  Tcl 96
% turned into a smartass generic template by De Clarke, with thanks to both 
% the above pioneers
%
% use at your own risk.  Complaints to /dev/null.
% make it two column with no page numbering, default is 10 point

% Munged by Fred Douglis <douglis@research.att.com> 10/97 to separate
% the .sty file from the LaTeX source template, so that people can
% more easily include the .sty file into an existing document.  Also
% changed to more closely follow the style guidelines as represented
% by the Word sample file. 

% Note that since 2010, USENIX does not require endnotes. If you want
% foot of page notes, don't include the endnotes package in the 
% usepackage command, below.

% This version uses the latex2e styles, not the very ancient 2.09 stuff.
\documentclass[letterpaper,twocolumn,10pt]{article}
    \usepackage{usenix,epsfig,endnotes,textcomp}
    \usepackage{hyperref}
    
    % TM - I added these packages to allow highlighting so "to do" items can stand
    % out and don't get forgotten.
    \usepackage{xcolor,soul}
    \usepackage{authblk}
    \usepackage{balance}
    
    \usepackage{upgreek}
    \usepackage{comment}
    
    \usepackage{geometry}
    
    \usepackage{tikz}
    \usetikzlibrary{calc}
    
    % GanttHeader setups some parameters for the rest of the diagram
    % #1 Width of the diagram
    % #2 Width of the space reserved for task numbers
    % #3 Width of the space reserved for task names
    % #4 Number of months in the diagram
    % In addition to these parameters, the layout of the diagram is influenced
    % by keys defined below, such as y, which changes the vertical scale
    \def\GanttHeader#1#2#3#4{%
     \pgfmathparse{(#1-#2-#3)/#4}
     \tikzset{y=7mm, task number/.style={left, font=\bfseries},
         task description/.style={text width=#3,  right, draw=none,
               font=\sffamily, xshift=#2,
               minimum height=2em},
         gantt bar/.style={draw=black, fill=blue!30},
         help lines/.style={draw=black!30, dashed},
         x=\pgfmathresult pt
         }
      \def\totalmonths{#4}
      \node (Header) [task description] at (0,0) {\textbf{\large Task Description}};
      \begin{scope}[shift=($(Header.south east)$)]
        \foreach \x in {1,...,#4}
          \node[above] at (\x,0) {\footnotesize\x};
     \end{scope}
    }
    
    % This macro adds a task to the diagram
    % #1 Number of the task
    % #2 Task's name
    % #3 Starting date of the task (month's number, can be non-integer)
    % #4 Task's duration in months (can be non-integer)
    \def\Task#1#2#3#4{%
    \node[task number] at ($(Header.west) + (0, -#1)$) {#1};
    \node[task description] at (0,-#1) {#2};
    \begin{scope}[shift=($(Header.south east)$)]
      \draw (0,-#1) rectangle +(\totalmonths, 1);
      \foreach \x in {1,...,\totalmonths}
        \draw[help lines] (\x,-#1) -- +(0,1);
      \filldraw[gantt bar] ($(#3, -#1+0.2)$) rectangle +(#4,0.6);
    \end{scope}
    }
    
    \begin{document}
    
    %make title bold and 14 pt font (Latex default is non-bold, 16 pt)
    \title{PERC: Persistent, Efficient, Recoverable, Consistent \\
           Research Proficience Evaluation Proposal
    }
    % \titlenote{Produces the permission block, and copyright information}
    % \subtitle{Utilizing Non-volatile Dual Inline Memory Modules}
    
    %for single author (just remove % characters)
    %\author{
    %{\rm Tony Mason}\qquad
    %{\rm Jake Wires}\qquad
    %{\rm Andrew Warfield}\\
    %{\rm fsgeek@cs.ubc.ca}\qquad
    %{\rm jtwires@cs.ubc.ca}\qquad
    %{\rm andy@cs.ubc.ca}\\
    %Department of Computer Science, University of British Columbia, Vancouver, Canada
    %} % end author
    
    \author{Tony Mason\thanks{fsgeek@cs.ubc.ca}}
    %\author{Jake Wires\thanks{jtwires@cs.ubc.ca}}
    %\author{Andrew Warfield\thanks{andy@cs.ubc.ca}}
    \affil{Department of Computer Science, University of British Columbia}
    
    % suppress printing date
    \date{}
    
    \maketitle
    
    \begin{abstract}
        This document is a supplement to my main \textbf{Research Proficiency Evaluation} proposal.  It is unlikely
        to ``make sense'' outside the context of that document.
  
    \end{abstract}
    
    \input{supplement}
    
    \newpage
    \balance
    
    {\footnotesize \bibliographystyle{acm}
    \bibliography{nvdimm.bib,phd-app-ref.bib}}
    
    
    % \theendnotes
    
    \end{document}
    
    