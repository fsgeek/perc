%% The following is a directive for TeXShop to indicate the main file
%%!TEX root = report.tex

\chapter{Related Work}
\label{ch:RelatedWork}

The field of non-volatile memory is one with a long history.  As noted in Chapter \ref{ch:Introduction}, 
early memories were, in fact, persistent and this led to models in which memory and storage were viewed
as equivalent storage mechanisms.~\cite{corbato1965introduction}  The introduction of dynamic ram (\acs{DRAM})
led to the bifurcation in storage technologies between high-speed but ephemeral (``memory'') and low-speed
but persistent (``media'').~\cite{US3728695A}

Our modern model of storage has evolved as well, from the punch cards of the late 19th century~\cite{truesdell1965development},
to more modern media based magnetic devices, such as tape (ca. 1935) and other forms of magnetic
media.~\cite{feynman1992there, hoagland2003history}

The introduction of flash memory~\cite{masuoka1984new} led to flash storage devices.  Progress in this area has led to the
emergence of solid state disk drives.~\cite{clay1995flash}  \acs{SSD}s today are often the primary storage device of numerous
computers, despite their higher cost than traditional hard disks, due to their performance characteristics.

The push to make \acs{SSD}s faster has led to a rapid progression of bus technologies that permit exploitation of the 
increasing bandwidth and decreasing latency of such devices.  The (re)-convergence of storage and memory has been demonstrated 
and discussed for more than 20 years.\cite{wu1994envy,mogul2009operating,Nalli:2017:APM:3093336.3037730,Nalli:2017:APM:3093337.3037730,
Nalli:2017:APM:3037697.3037730,kolli2017architecting,joshi2015efficient}  
The advent of \acs{PCM} was one of those promising
technologies.~\cite{chen2016review}  The research community has explored this area extensively in the ensuing years in such topics as:

\begin{itemize}

    \item Memory Management:\cite{volos2011mnemosyne,Oukid:2017:MMT:3137628.3137629,yu2017redesign}
    \item Data Structures:\cite{Friedman:2018:PLQ:3178487.3178490,Memaripour:2017:AIU:3064176.3064215,lee2017wort,sha2018towards,zhang2018simpo,
kim2018clfb,wang2018persisting,nawab2017dali,lee2018write,oukid2017data,venkataraman2011consistent,chen2015persistent,222591,
zuo2017write,zuo2018write}
    \item Multi-threading: \cite{Hsu:2017:NPP:3064176.3064204}
    \item Programming: \cite{Kolli:2017:LP:3079856.3080229,Kolli:2017:LP:3140659.3080229,marathe2017persistent}
    \item File Systems:\cite{kannan2018designing,liu2018hmfs}
    \item Logging:\cite{Cohen:2017:ELN:3152284.3133891,shin2017proteus}
    \item Crash Consistency:\cite{Pillai:2017:ACC:3141876.3119897,wei2017transactional}
    \item Address Translation:\cite{Chen:2017:ESP:3123939.3124543,liu2017durable,wang2017hardware}
    \item Database:\cite{arulraj2017build,Oukid:2014:SHS:2619228.2619236,arulraj2015let}
    \item Checkpointing: \cite{zhang2017fine,giles2017continuous}
    \item Key-Value Stores:\cite{huang2018nvht,chen2017udorn,kim2017papyruskv,liu2017librekv,zhou2016nvht,wu2016nvmcached}
    \item Transations:\cite{marathe2018persistent}
\end{itemize}

Prior work has been done with a variety of models for simulating the behavior of \acs{NVM}.  Indeed, my
initial investigation was to determine the viability of investigating some of the considerations described
in Chapter \ref{ch:Introduction} via simulation.  Ultimately, I was able to gain access to an Intel Apache
Pass research system for performing this research.

Thus, one key difference between the prior work and the work described in this report is the use of actual
hardware.  As expected, that hardware does not behave as predicted.  I will discuss this further in Chapters \ref{ch:Results}
\& \ref{ch:Discussion}. 

\endinput

