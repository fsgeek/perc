%% The following is a directive for TeXShop to indicate the main file
%%!TEX root = report.tex

\chapter{Abstract}

Memory --- the ability to save and recall information --- is a fundamental
characteristic of human endeavour and takes many forms.  As we developed
computing machines, we similarly developed mechanisms by which information
could be stored for later use.

For modern computers, \textit{drum memory} was the first manifestation of
magneto-electric data storage and at the time of its introduction was used
as both working memory as well as longer-term storage.  Drum memory was
replaced as working memory by \textit{core memory}, which utilized magnetic
wrapped cores for storing bits of information.  Similarly core memory was
in turn replaced by \acs{DRAM}.  

Each new class of memory exhibited faster performance but \textit{different} 
behavior than the previous class.  Drum and core memories were persistent,
but core memory had a destructive write phase.  \acs{DRAM} memory was not
persistent and required constant refresh to prevent the contents from
decaying.

While faster, \acs{DRAM} abandoned persistence of memory and gave rise to the
separation of \textit{memory} and \textit{storage}.  \acs{DRAM} memories and
processors became faster at a more rapid rate than storage became faster,
further increasing the separation.  While \acs{SRAM}
is faster than \acs{DRAM}, it is much more expensive and only persistent as long as
power is applied to it.

For decades, researchers have been searching for a new memory technology 
that is comparable in terms of performance and behavior to DRAM but
\textit{also} persistent.  This achievement has proven to be elusive yet
that has not discouraged the research community from considering how
to exploit persistent byte-addressable non-volatile memory.

This report describes my findings while using the first commercial product
to offer single level, byte-addressable non-volatile computer memory that behaves much like
\acs{DRAM} and observes how this behavior might impact development of
systems that exploit this ``new'' class of memory.

% Consider placing version information if you circulate multiple drafts
\vfill
\begin{center}
\begin{sf}
\fbox{Revision: \today}
\end{sf}
\end{center}
